\documentclass[a4paper,12pt,oneside,openany]{jsbook}
\bibliographystyle{junsrt}

%%%%%%%%%%%%%%%%%%%利用パッケージ一覧%%%%%%%%%%%%%%%%%%%
\usepackage[T1]{fontenc}       %Latin Modern
\usepackage{lmodern}           %Latin Modern
\usepackage{amsmath,amssymb}  % 数式用
\usepackage{bm}               % 太文字ベクトル用
\usepackage[dvipdfmx, nosetpagesize]{graphicx} %画像用
\usepackage{subcaption}
\usepackage{verbatim}
\usepackage{wrapfig}
\usepackage{ascmac}
\usepackage{makeidx}
\usepackage{fancyhdr}
\usepackage{layout}
\usepackage{caption}  %`キャプションのスタイル用
\usepackage{booktabs} %for table
\usepackage{float}
\usepackage{multienum} %2段組
\usepackage{url}       %url
\usepackage[a4paper,truedimen]{geometry} %上下左右の余白調整
%%%%%%%%%%%%%%%%%%%ドキュメントスタイル%%%%%%%%%%%%%%%%%%%%
\setcounter{tocdepth}{2} %サブセクションまで目次に表示する
%\makeindex
\geometry{left=30mm, right=30mm, top=35mm, bottom=30mm} %上下左右の余白調節
\renewcommand{\baselinestretch}{0.9} % 行間の調整
\renewcommand\UrlFont{\rmfamily}     % urlフォントの変更

%%%% ヘッダー・フッターの設定 %%%%
\pagestyle{fancy}
\lhead{}
\rhead{\thepage}
\cfoot{}
\renewcommand{\headrulewidth}{0pt}

%%%% サイトスタイルの設定 %%%%
\makeatletter 
\renewcommand{\@cite}[1]{\textsuperscript{\,[#1]}} 
\makeatother

%%%%%%%%%%%%%%%%%%ドキュメント開始%%%%%%%%%%%%%%%%%%%%%
\begin{document}
\frontmatter
%目次のカウント開始
% 目次のページだけページ番号付けない方法
% https://tex.stackexchange.com/questions/2995/removing-page-number-from-toc
\addtocontents{toc}{\protect\thispagestyle{empty}}
\tableofcontents
\thispagestyle{empty}

%本文開始
\mainmatter

%%%%%%%%%%%%%%%%%%%%%%本文%%%%%%%%%%%%%%%%%%%%%%%%%
% 1.序論
\include{./1_intro/1_main}

% 2.関連研究
\include{./2_related_works/2_main}

% 3.提案手法
\include{./3_proposed_method/3_main}

% 4.実験
\include{./4_experiment/4_main}

% 5.考察
\include{./5_consideration/5_main}

% 6.結論
\include{./6_conclusion/6_main}

% 引用
%% !TEX root = ../main.tex
%章番号を付けない場合は,chapter*{}とする.
% https://home.hirosaki-u.ac.jp/masumi/100/
\chapter*{謝辞}
% https://tex.stackexchange.com/questions/35433/creating-unnumbered-chapters-sections-plus-adding-them-to-the-toc-and-or-header
\addcontentsline{toc}{chapter}{謝辞}
\thispagestyle{fancy} %ヘッダー・フッターの設定
めちゃありがとう

\bibliography{ref}
\thispagestyle{fancy}

%添付資料
% !TEX root = ../main.tex

%\appendix
%\chapter{aaaaaa}

%iiiiiiii


% 謝辞
% !TEX root = ../main.tex
%章番号を付けない場合は,chapter*{}とする.
% https://home.hirosaki-u.ac.jp/masumi/100/
\chapter*{謝辞}
% https://tex.stackexchange.com/questions/35433/creating-unnumbered-chapters-sections-plus-adding-them-to-the-toc-and-or-header
\addcontentsline{toc}{chapter}{謝辞}
\thispagestyle{fancy} %ヘッダー・フッターの設定
めちゃありがとう


\end{document}
